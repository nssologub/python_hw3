го состояния зависит от s
прошлых, является цепь Маркова s-го порядка. Однако число независимых
параметров полносвязной цепи Маркова s-го порядка возрастает экспоненциально с
увеличением s, поэтому использование этой модели требует больших затрат памяти. Для решения этой проблемы разрабатываются так называемые «малопараметрические» модели цепи Маркова s-го порядка. Примерами таких моделей являются цепь Маркова с частичными связями, модель Рафтери, цепь Маркова условного порядка, цепь Маркова переменного порядка

Отметим также, что векторная цепь Маркова с частичными связями не может рассматриваться как семейство одномерных цепей Маркова из-за наличия зависимости между компонентами вектора в модели $VMC(s, M_r)$.

\subsection{Цель и задачи работы}
Целью данной  работы является разработка алгоритмов и программных средств статистического оценивания параметров векторной цепи Марков а с частичными связями. Для достижения этой цели необходимо решить следующие задачи:
\begin{enumerate}
\item Разработать математическую модель VMC(s, $M_r$);
\item Разработать компьютерную модель для имитации реализации VMC(s, $M_r$), провести тестирование компьютерной модели;
\item Разработать алгоритмы статистического оценивания параметров модели: матрицы вероятностей одношаговых переходов, шаблона связей, количества связей и порядка цепи Маркова;
\item Реализовать алгоритмы статистического оценивания параметров на компьютере и провести компьютерные экперименты по оценке их точности и быстродействия.
\end{enumerate}
